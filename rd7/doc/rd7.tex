%%%%%%%%%%%%%%%%%%%%%%%%%%%%%%%%%%%%%%%%%%%%%%%%%%%%%%%%%%%%%%%%%%%%%%%%%%%%
% 
% rd7.tex -- Macro file for rd7 documents
%
%%%%%%%%%%%%%%%%%%%%%%%%%%%%%%%%%%%%%%%%%%%%%%%%%%%%%%%%%%%%%%%%%%%%%%%%%%%%
 
%
% Useful Macros
%
\newcommand{\nl}{\mbox{}\\\mbox{}}		% for newline after item tag
\newcommand{\code}[1]{\mbox{\tt #1}}
\newcommand{\param}[1]{\mbox{\it #1}}
\newcommand{\key}[1]{{\sc #1}}
\newcommand{\cmd}[2]{`#1' (#2)}
\newcommand{\ignore}[1]{}

\newcommand{\Fref}[1]{Figure \ref{#1}}

\newenvironment{dcode}{\begin{verse} \tt }{\end{verse}}

\hyphenation{back-space}

%%%%%%%%%%%%%%%%%%%%%%%%%%%%%%%%%%%%%%%%%%%%%%%%%%%%%%%%%%%%%%%%%%%%%%%%%%%%
%
% Logos and Abbreviations
%
%%%%%%%%%%%%%%%%%%%%%%%%%%%%%%%%%%%%%%%%%%%%%%%%%%%%%%%%%%%%%%%%%%%%%%%%%%%%

% This is the basic large (40/80pt) line-drawing logo
%
\newcommand{\logo}{{\thicklines
 \begin{picture}(92,40)(-2,0)
%				% r
 \put(0,0){\line(1,2){10}}
 \put(10,20){\line(1,0){10}}
%
 \put(-2,0){\line(1,2){11}}
 \put(9,22){\line(1,0){13}}
%				% d
 \put(20,0){\line(1,0){20}}
 \put(20,0){\line(1,2){10}}
 \put(30,20){\line(1,0){20}}
 \put(40,0){\line(1,2){20}}
%
 \put(23,2){\line(1,0){17}}
 \put(23,2){\line(1,2){8}}
 \put(31,18){\line(1,0){17}}
 \put(42,0){\line(1,2){20}}
%				% 7
 \put(70,38){\line(1,0){19}}
 \put(70,0){\line(1,2){19}}
%
 \put(72,40){\line(1,0){20}}
 \put(72,0){\line(1,2){20}}
 \end{picture}
}}

% This is the basic 20-pt line-drawing logo
%
\newcommand{\llogo}{{\thicklines
 \begin{picture}(47,20)
%				% r
 \put(0,0){\line(1,2){5}}
 \put(5,10){\line(1,0){5}}
%				% d
 \put(12,0){\line(1,0){10}}
 \put(12,0){\line(1,2){5}}
 \put(17,10){\line(1,0){10}}
 \put(22,0){\line(1,2){10}}
%				% 7
 \put(37,20){\line(1,0){10}}
 \put(37,0){\line(1,2){10}}
 \end{picture}
}}

% This is a logo 10 pts high.  It is kerned slightly.  
% Note the kludgery required because LaTeX won't draw short diagonal
% lines.  Bezier actually scales correctly under magnification
% (though some fonts don't exist at some mags)
%
\newsavebox{\sslant}
\savebox{\sslant}(2.5,5){\bezier{20}(0,0)(1,2)(2.5,5)}
\newsavebox{\lslant}
\savebox{\lslant}(5,10){\bezier{40}(0,0)(1,2)(5,10)}
\newcommand{\slogo}{{
 \begin{picture}(20,10)
%				% r
 \put(0,0){\usebox\sslant}
 \put(2.5,5){\line(1,0){3}}
%				% d
 \put(6,0){\line(1,0){5}}
 \put(6,0){\usebox\sslant}
 \put(8.5,5){\line(1,0){5}}
 \put(11,0){\usebox\lslant}
%				% 7
 \put(18.5,10){\line(1,0){5}}
 \put(18.5,0){\usebox\lslant}
 \end{picture}
}}

\newcommand{\seven}{{\setlength{\unitlength}{1pt}\begin{picture}(5,10)
 \put(0,0){\line(1,2){5}}
 \put(0,10){\line(1,0){5}}
\end{picture}}}

%
% These are the ``standard'' versions of rd7, ab, and sp
%
%\newcommand{\RD}{\mbox{\Large\bf\sf rd\hspace{2pt}\seven}}
\newcommand{\RD}{\slogo}
\newcommand{\AB}{\mbox{\large\bf\sl ab}}
\newcommand{\SP}{\mbox{\large\bf\sl sp}} 
