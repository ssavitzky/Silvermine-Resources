\documentstyle[bezier,12pt]{book}
\pagestyle{headings}

\setlength{\unitlength}{1.2pt}
\thicklines
%%%%%%%%%%%%%%%%%%%%%%%%%%%%%%%%%%%%%%%%%%%%%%%%%%%%%%%%%%%%%%%%%%%%%%%%%%%%
% 
% rd7.tex -- Macro file for rd7 documents
%
%%%%%%%%%%%%%%%%%%%%%%%%%%%%%%%%%%%%%%%%%%%%%%%%%%%%%%%%%%%%%%%%%%%%%%%%%%%%
 
%
% Useful Macros
%
\newcommand{\nl}{\mbox{}\\\mbox{}}		% for newline after item tag
\newcommand{\code}[1]{\mbox{\tt #1}}
\newcommand{\param}[1]{\mbox{\it #1}}
\newcommand{\key}[1]{{\sc #1}}
\newcommand{\cmd}[2]{`#1' (#2)}
\newcommand{\ignore}[1]{}

\newcommand{\Fref}[1]{Figure \ref{#1}}

\newenvironment{dcode}{\begin{verse} \tt }{\end{verse}}

\hyphenation{back-space}

%%%%%%%%%%%%%%%%%%%%%%%%%%%%%%%%%%%%%%%%%%%%%%%%%%%%%%%%%%%%%%%%%%%%%%%%%%%%
%
% Logos and Abbreviations
%
%%%%%%%%%%%%%%%%%%%%%%%%%%%%%%%%%%%%%%%%%%%%%%%%%%%%%%%%%%%%%%%%%%%%%%%%%%%%

% This is the basic large (40/80pt) line-drawing logo
%
\newcommand{\logo}{{\thicklines
 \begin{picture}(92,40)(-2,0)
%				% r
 \put(0,0){\line(1,2){10}}
 \put(10,20){\line(1,0){10}}
%
 \put(-2,0){\line(1,2){11}}
 \put(9,22){\line(1,0){13}}
%				% d
 \put(20,0){\line(1,0){20}}
 \put(20,0){\line(1,2){10}}
 \put(30,20){\line(1,0){20}}
 \put(40,0){\line(1,2){20}}
%
 \put(23,2){\line(1,0){17}}
 \put(23,2){\line(1,2){8}}
 \put(31,18){\line(1,0){17}}
 \put(42,0){\line(1,2){20}}
%				% 7
 \put(70,38){\line(1,0){19}}
 \put(70,0){\line(1,2){19}}
%
 \put(72,40){\line(1,0){20}}
 \put(72,0){\line(1,2){20}}
 \end{picture}
}}

% This is the basic 20-pt line-drawing logo
%
\newcommand{\llogo}{{\thicklines
 \begin{picture}(47,20)
%				% r
 \put(0,0){\line(1,2){5}}
 \put(5,10){\line(1,0){5}}
%				% d
 \put(12,0){\line(1,0){10}}
 \put(12,0){\line(1,2){5}}
 \put(17,10){\line(1,0){10}}
 \put(22,0){\line(1,2){10}}
%				% 7
 \put(37,20){\line(1,0){10}}
 \put(37,0){\line(1,2){10}}
 \end{picture}
}}

% This is a logo 10 pts high.  It is kerned slightly.  
% Note the kludgery required because LaTeX won't draw short diagonal
% lines.  Bezier actually scales correctly under magnification
% (though some fonts don't exist at some mags)
%
\newsavebox{\sslant}
\savebox{\sslant}(2.5,5){\bezier{20}(0,0)(1,2)(2.5,5)}
\newsavebox{\lslant}
\savebox{\lslant}(5,10){\bezier{40}(0,0)(1,2)(5,10)}
\newcommand{\slogo}{{
 \begin{picture}(20,10)
%				% r
 \put(0,0){\usebox\sslant}
 \put(2.5,5){\line(1,0){3}}
%				% d
 \put(6,0){\line(1,0){5}}
 \put(6,0){\usebox\sslant}
 \put(8.5,5){\line(1,0){5}}
 \put(11,0){\usebox\lslant}
%				% 7
 \put(18.5,10){\line(1,0){5}}
 \put(18.5,0){\usebox\lslant}
 \end{picture}
}}

\newcommand{\seven}{{\setlength{\unitlength}{1pt}\begin{picture}(5,10)
 \put(0,0){\line(1,2){5}}
 \put(0,10){\line(1,0){5}}
\end{picture}}}

%
% These are the ``standard'' versions of rd7, ab, and sp
%
%\newcommand{\RD}{\mbox{\Large\bf\sf rd\hspace{2pt}\seven}}
\newcommand{\RD}{\slogo}
\newcommand{\AB}{\mbox{\large\bf\sl ab}}
\newcommand{\SP}{\mbox{\large\bf\sl sp}} 


\title{\logo \\ User's Guide} 
\author{Copyright \copyright\ 1988 by Silvermine Resources, Inc.\\
	All rights reserved.}

%
% Page Parameters (multiplied by 1.2 for 12pt---works better than magstep1)
%
\oddsidemargin 0in \evensidemargin 0in \topmargin 0in
\columnsep 10pt \columnseprule 0pt 
\marginparwidth 90pt \marginparsep 11pt \marginparpush 5pt 
\ignore{ 
 \headheight 12pt \headsep 36pt 
 \footheight 12pt \footskip 25pt 
 \textheight 6.5in \textwidth 4in
}
\headheight 14.4pt \headsep 43.2pt 
\footheight 14.4pt \footskip 28.8pt 
\textheight 7.8in \textwidth 4.8in

%%%%%%%%%%%%%%%%%%%%%%%%%%%%%%%%%%%%%%%%%%%%%%%%%%%%%%%%%%%%%%%%%%%%%%%%%%%%
 
\begin{document}

\titlepage
{
\setlength{\unitlength}{2.4pt}
\thicklines
\vspace{1in}
\begin{center}

\newsavebox{\lsslant}
\savebox{\lsslant}(2.5,5){\bezier{20}(0,0)(1,2)(2.5,5)}
\newsavebox{\llslant}
\savebox{\llslant}(5,10){\bezier{40}(0,0)(1,2)(5,10)}
\newcommand{\lslogo}{{
 \begin{picture}(20,10)
%				% r
 \put(0,0){\usebox\lsslant}
 \put(2.5,5){\line(1,0){3}}
%				% d
 \put(6,0){\line(1,0){5}}
 \put(6,0){\usebox\lsslant}
 \put(8.5,5){\line(1,0){5}}
 \put(11,0){\usebox\lslant}
%				% 7
 \put(18.5,10){\line(1,0){5}}
 \put(18.5,0){\usebox\llslant}
 \end{picture}
}}

\begin{picture}(140,85)
\put(1,1){\framebox(138,83){}}
\put(3,67){\framebox(134,15)[l]{\lslogo}}
\put(3,67){\makebox(134,15)[r]{\huge ab/sp\ }}
\put(3,3){\framebox(134,62)[t]{\makebox(134,50){\logo}}}
\put(3,3){\makebox(134,62)[b]{\makebox(134,16){\Huge User's Guide}}}
\end{picture}

\copyright\ 1988 by Silvermine Resources, Inc.\quad\quad
All rights reserved.

\end{center}
}


\tableofcontents
\markboth{}{Contents}

% === need blank page at end of TOC ===

\chapter{Introduction}
 
\RD\ is a program that runs on the IBM PC, XT, AT, and compatible
computers, for reading diskettes written on the Perkin-Elmer 7000 series
Professional Data Stations under the Idris operating system,
and transferring them to PC/MS-DOS diskettes or hard disks. 


\section{Features}
 
\begin{itemize}
\item
A display-oriented presentation of the directory structure of both
source (Idris) and destination (DOS) file systems.  The display
includes multiple views including side-by-side display of a disk's
directory structure (in tree form) and the contents of a selected
directory in the tree.
 
\item
The ability to view the contents of files in either ASCII or
Hexadecimal on either the source or destination file system.
 
\item
Menus with extensive context-sensitive help.
 
\item
The ability to transfer an entire directory tree, preserving its
structure, with only two keystrokes.
 
\end{itemize}

\newpage
\section{Versions}
 
\RD\ comes in two versions:
 
\begin{enumerate}
 
\item \RD-AB for ASCII and Binary files.
 
\item \RD-SP, which in addition to ASCII and binary data
can read spectral data in Perkin-Elmer's
\code{.sp} format, view spectra in graphical form, and copy spectra
either directly, translated to JCAMP-DX form, or translated to
Perkin-Elmer's new Data Manager format for PC-based computers.
 
\end{enumerate}
 
\noindent This manual describes both versions.
 
 
\chapter{Installation}
 
\section{Disk Contents}
The \RD\ disk contains the following files:
 
\begin{description}
\item[\code{RD7.EXE}] \nl
The executable file that implements the \code{rd7} command.  
 
\item[\code{AUTOEXEC.BAT}] \nl
A sample file that can be appended to your own
\code{AUTOEXEC.BAT} file to change default parameters for \RD.
 
\item[\code{JCAMP.HDR}] \nl
{\em [\RD-SP only]} A sample Jcamp header file.

\item[\code{MSHERC.COM}] \nl
{\em [\RD-SP only]} A device driver that lets you use \RD-SP with a
Hercules (or compatible) monochrome graphics card.
 
\item[\code{README}] \nl
A text file containing last-minute notes.
 
\end{description}
 
\newpage 
\section{Installing Files}

\begin{enumerate} 

\item
Start by making a {\em copy} of your \RD\ disk.
On a system with two floppy drives, give the command
\begin{dcode}
diskcopy a: b:
\end{dcode}
\noindent
and then put the \RD\ disk in drive \code{A:}, and a blank disk in
drive \code{B:}.
 
On a system with only one floppy drive, give the command
\begin{dcode}
diskcopy
\end{dcode}
\noindent
and then put the \RD\ disk in the floppy drive.  Exchange it with a
blank disk when the program prompts you to do so.

\item 
Put the original disk in a safe place.

\item 
If you have a hard disk, make a directory on your working disk
(usually \code{C:}) called \code{RD7}, and copy all of the files on
the \RD\ disk to it.  You can do this with the commands:
\begin{dcode}
mkdir rd7\\
cd rd7\\
copy a:*.*
\end{dcode}
\noindent (assuming that the \RD\ disk is in drive \code{A:}).
 
(Strictly speaking, you don't have to do this.  You can put \RD\
in any directory that is listed in the \code{PATH} command in your
\code{AUTOEXEC.BAT} file.  We find it convenient, however, to keep
\RD\ and all of our 7000 files in one place; the examples below
will assume this configuration.)
 
If you will be using floppy disks, you will probably find it best to
make another copy of the \RD\ disk and switch to it after booting
from your usual system disk.  It is also possible to put
\code{RD7.EXE} on your system disk and switch to a blank (but
formatted) DOS disk after starting \RD.  This leaves your entire
DOS disk free for transferring files.

If you are using \RD-SP on a machine with a Hercules or compatible
graphics card, you will need to run \code{MSHERC} before running \RD.
\code{MSHERC} is a ``Terminate-and-Stay-Resident'' device driver that
allows \RD-SP's graphics displays to operate on the Hercules card.
The simplest way to ensure that \code{MSHERC} is run is to put a line
containing just ``\code{MSHERC}'' into your \code{AUTOEXEC.BAT} file,
which is executed whenever you boot your machine.  The sample
\code{AUTOEXEC.BAT} file on your \RD-SP distribution disk contains this
line.

\end{enumerate}

\section{Getting Started}

Finally, give the command:
\begin{dcode}
rd7
\end{dcode}
\noindent
to start the \RD\ program.  The screen will look like \Fref{fig-startup}.
 
\begin{figure}[tbp]

\vspace{1.25in}
((( TO BE SUPPLIED )))
\vspace{1.25in}
 
\caption{Startup Screen}
\label{fig-startup}
\end{figure}


 
A menu is shown on the second line of the screen.  You can issue a
command from the menu either by typing the first character of the
command, or by moving the highlight to the command (\key{space} moves
forward and \key{backspace} moves backward) and hitting the \key{enter}
key.  The `?' key will {\em always} show you a screen with a
short summary of the current menu and its commands; the `/' key
always brings you back to the initial menu.  The \cmd{A}{About}
command in the initial menu gives you an overview of \RD\ and its
commands.
 
Try moving the highlighted menu item with the \key{space} and
\key{backspace} keys.  Notice that the line under the menu contains a
brief description of the highlighted item.  (The first word in the
menu is not a command but a label for the menu, and its help line
gives a general description of what commands in that menu are for.)

The line below the description line contains labels for the views
that occupy the lower portion of the screen.  On the left end of the
label line is the name of the view that will be reached by striking
the left arrow key (``Dos'') in the initial view, and on the right is
the name of the view that will be reached with the right arrow key.%
\footnote{In case you were wondering why double-angle symbols are
shown on the screen instead of arrows, it's because you might want to
use the ``Print Screen'' key, and the arrows correspond to control
characters that most printers can't handle.}  When there are two
views on the screen at the same time, the arrow labels will appear
over the ``active'' view, i.e. the one the menu refers to.

Next, try giving the commands \cmd{D}{DOS} (or left arrow) 
and \cmd{R}{Read Disk}.
(Commands are single keystrokes; no \key{enter} key is needed, and
either upper- or lower-case letters will work.)  You will end up with
a display that looks like \Fref{fig-after-read}.
 
\begin{figure}[tbp]

\vspace{1.25in}
((( TO BE SUPPLIED )))
\vspace{1.25in}
 
\caption{After the commands `D' and `R' the display will look
like this.}
\label{fig-after-read}
\end{figure}
 
Use the right arrow key to move the highlight into the File view. 
Highlight the file called \code{README} by moving down with the 
down arrow key.  Use the right arrow key again to view the contents
of the file.  Use the left arrow key twice to get back
into the Directory View.
 
Now put a 7000 Idris disk in drive \code{A:} and give the commands
left-arrow (or \cmd{/I}{Idris}) and \cmd{R}{Read}.  (If you are working from
floppies and have the \RD\ disk in drive \code{A:}, put the Idris
disk in drive \code{B:} and give the commands \cmd{I}{Input}, \cmd{D}{Drive},
and `1'.  Notice that when you are in the Idris views,
the drives are designated by numbers the way they are on the 7000,
rather than letters as they are in DOS.)  In any case, you will
get a display something like \Fref{fig-idris}.
 
\begin{figure}[tbp]
 
\vspace{1.25in}
((( TO BE SUPPLIED )))
\vspace{1.25in}
 
\caption{After the commands `/', `I' and `R' the display will look
something like this.}
\label{fig-idris}
\end{figure}
 
To transfer the entire contents of the Idris disk to the DOS disk,
simply give the commands \cmd{T}{Tag} and \cmd{W}{Write}.
If you want to be more selective, you can select
individual directories in the Directory view, or files in the File
view, and tag them with the `T' command.  The `W' command in the
File view simply writes the highlighted file without tagging it; in
the Directory view it writes all tagged files in the highlighted
directory.
 
Note that by default, all files except spectra (with a \code{.sp}
extension) are assumed to be ASCII, and any newline characters they
contain are converted to carriage-return/linefeed pairs.  You can
designate files as Binary (and tag them for transfer at the same time)
with the \cmd{B}{Binary} command.
Giving these commands
%The \cmd{C}{Conversion} command
in the Directory View lets you tag all the files under a directory and
also designate a conversion type for them.
 

\chapter{Using rd7}
 
\section{Using the Menus}
 
Menus are shown on the second line of the screen.  You can issue a
command from the menu either by typing the first character of the
command, or by moving the highlight to the command (\key{space} moves
forward and \key{backspace} moves backward) and hitting the \key{enter}
key.  When using the first character of the command, only a single
keystroke is needed (no \key{enter} key), and either upper- or
lower-case can be used for letters.
 
To find out more about a menu command, use \key{space} and
\key{backspace} to move the highlight to that
command.  The line underneath the menu gives a brief description of
the command.  To find out more, type the `?' key, which will always
show you a screen with a short summary of the current menu and its
commands; the `/' key always brings you back to the initial menu. 
The \cmd{A}{About} command in the initial menu gives a more
extensive overview of \RD\ and its commands.
 
 
\section{Using the Cursor Keys}
 
The cursor keys are used for moving within a view, and between views. 
For moving within a view, the up and down arrows move a line at a
time, and the \key{PgUp} and \key{PgDn} keys move a screen-full at a
time.  The \key{Home} key moves to the first item in the view (i.e.
the root directory, first file in a directory, or first line in a
file).  The \key{End} key moves to the last item in the view.
 
The left and right arrow keys are used to move {\em between} views.
If there are two views on the screen, (e.g. the Directory and File
views), left and right arrows will move between them in the obvious
fashion.  Moving left from the Directory view goes to the {\em other}
Directory view (i.e. Idris from DOS, and DOS from Idris). 
The names of the views reached with the left and right arrows are
listed at the left and right edges of the screen (or active view, if
two views are on the screen), on the top line of the view frame.

Moving to the right always gets you more detail, and moving left less
(or something completely different, as in switching between Directory
Views).  Thus, moving right from the Directory view gets you to the
File view and moving right from the File view shows you the {\em
contents} of the highlighted file.  If you are looking at a help
screen (the result of the `?' command), moving right gets you help
about the help menu, and moving left gets you back to your previous
view.
 
 
\section{Transferring Files}
 

Transferring files is a three-step process:

\begin{enumerate}
\item 
Select a destination directory for the files, using the \cmd{O}{Output}
command in the DOS Directory View.

\item 
Tag some files to transfer (and possibly specify a conversion type for
them as well).  The \cmd{T}{Tag} command tags the highlighted
file in the File View, and {\em everything} under the highlighted
directory in the Directory View.  The \cmd{A}{ASCII} and \cmd{B}{Binary}
commands in the File View both tag and specify a
conversion type, as do \RD-SP's \cmd{S}{Spectrum}
and \cmd{J}{Jcamp} commands.  

You can assign a conversion type to everything under a directory using
the conversion type commands in the Directory View.

\item
Write the files with the \cmd{W}{Write} command in the
Directory View.

\end{enumerate}

\noindent
When transferring a single file the second step can be omitted by
issuing the \cmd{W}{Write} command in the File View with the
file you want to transfer highlighted.

For Jcamp conversion, you must also select a Jcamp header file using
the \cmd{J}{Jcamp} command in the DOS File View.

When a file is written its name may have to be changed; this is done
automatically.  A file's name may be changed in the following ways:
\begin{itemize}
\item
DOS filenames can contain only uppercase letters and a restricted set
of special characters.  Lowercase letters in the Idris filename are
uppercased, and special characters not permitted in DOS are replaced
by `\code{~}' characters.
\item
DOS filenames consist of an eight-character name, a period, and a
three-character extension.  The name and extension of the Idris file
are truncated if necessary.
\item
If the file to be written already exists, a number is appended to its
name (or replaces the last characters of the name, if the name is
long enough).  The number is incremented until a name conflict no
longer occurs.
\item
In \RD-SP, Spectrum files (\code{.sp} extension) written in Binary
mode have their extension changed to \code{.SPI}; spectra written in
JCAMP-DX mode have their extension changed to \code{.DX}.
\end{itemize}

You can transfer files from DOS as well as Idris.  This is
slightly less convenient when transferring between disks because you
have to use the \cmd{D}{Drive} command to switch between source and
destination disks rather than the arrow keys.  Note, however, that \RD
keeps a separate set of tagged files for each disk you have read.
This means that, if necessary, you can tag some files, select another
disk, specify an Output directory, go back to the first disk, and
write the files you tagged.

Note that you {\em cannot} transfer files {\em to} an Idris
disk---\RD\ doesn't know how to write Idris files, only read them.

If you decide that you need a new subdirectory to transfer your files
to, the \cmd{N}{New} command in the DOS Directory view lets you enter
the name of a directory to be created (under the currently-highlighted
directory).  The name is entered on the prompt line (the line just under
the menu); backspace can be used to correct mistakes.

 
\section{Customization}
 
\RD\ has three parameters that can be customized:  the default disk
drives for Idris and DOS, and the default JCAMP header file.
These can be specified in either of two ways: in the \RD\ command line,
or in your \code{AUTOEXEC.BAT} file.

To specify customization parameters in the command line,
follow the command \code{rd7} with one or more of the following
options:

\begin{description}

\item[\code{-i} \param{drive-number}] \nl
This specifies the input (Idris) drive.  The drive should be specified
as a decimal number.

\item[\code{-o} \param{drive-letter}] \nl
This specifies the output (DOS) drive.  The drive should be specified
as a letter followed by a colon.  An additional directory path may be
appended; it will be set as your current working directory.  This lets
you start up \RD\ and have it \code{cd} to whatever directory you
usually transfer files to.

\item[\code{-j} \param{pathname}] \nl
This specifies the pathname of the default JCAMP header file.

\end{description}
\noindent Note that a space is {\em required} between the option
letter (i, o, or j) and the drive-letter, drive-number, or pathname
that follows it.

To specify customization parameters in your \code{autoexec.bat} file,
include a line of the form:
\begin{quote}
\code{set RD7 =} \param{options}
\end{quote}

\noindent where the options are the same as the command-line options
described above.


\chapter{Views and Operations}
 
This section contains a complete description of \RD's views and the
operations that can be performed in them (by means of both cursor keys
and menus).  It is a slightly expanded version of the online
help screens.


 
\section{Main View}

\begin{description}
\item[About] \nl
            This command gives you an overview of the program and
            how to use it.
\item[DOS] $(\leftarrow)$ \nl
            Transfer control to the DOS Directory View.
\item[Idris] $(\rightarrow)$ \nl
            Transfer control to the Idris Directory View.
\item[Quit] \nl
            Exit the program.
\item[?] \nl
            Display help.
\end{description}

 
\section{Drive Menu}

\begin{description}
\item[A-F] [DOS]
\item[0-7] [Idris]\nl
        Select a drive for input or output.  The directory of the
        drive you select will be read in and displayed in tree form.
\item[/ ] \nl
            The '/' command returns control to the Main View.
\item[Esc] \nl
            The Esc key returns to the Directory View with no action.

\end{description}

\ignore{
\section{Conversion Type Menu}

\begin{description}
\item[ASCII, Binary, Jcamp, Spectrum] \nl
            Specify the conversion type on ALL files in the current
            directory and its subdirectories, and TAG the files for
            output as well. 
\item[/] \nl
            The '/' command returns control to the Main View.
\item[Esc] \nl
            The Esc key returns to the Directory View with no action.
\end{description}
}

\section{Idris Directory View}

\begin{description}
\item[ASCII, Binary, Jcamp, Spectrum] \nl
%\item[Conversion] \nl
%            The Conversion command enters a sub-menu that lets you
            Specify the conversion type (A, B, D, J, S) for the entire
            directory and its subdirectories.  All files are tagged.
\item[Tag] \nl
            The Tag command marks the entire contents of the current
            directory, and all of its sub-directories, to be written
            out when the Write command is issued.
\item[Untag] \nl
            The Untag command removes the tag mark on the entire contents
            of the current directory and all of its sub-directories.
\item[Drive] \nl
            Select a disk drive for input and read its directory.
\item[Read] \nl
            Read the directory on the selected input (Idris) disk,
\item[Write] \nl
            Write all tagged files under the current directory.  If the
	    current directory has any subdirectories containing tagged
files, corresponding subdirectories will be created in the output
directory.
\item[Quit] \nl
            The Quit command exits the program.
\item[/] \nl
            The '/' command returns control to the Main View.
\end{description}


\section{Idris File View}

\begin{description}
\item[ASCII, Binary, Jcamp, Spectrum] \nl
            Tells what kind of conversion to do on the current file
            when it is written out.  (These also tag the file.)
\item[Tag] \nl
            The Tag command marks current file to be written out
            when the Write command is issued.
\item[Untag] \nl
            The Untag command removes the tag mark on current file.
\item[View] $(\rightarrow)$ \nl
            View a file in a window (non-ASCII files are viewed in Hex).
\item[Write] \nl
            Write the current file (whether or not it is tagged).
\item[Quit] \nl
            The Quit command exits the program.
\item[/] \nl
            The '/' command returns control to the Main View.
\end{description}


\section{DOS Directory View}

\begin{description}
 
\item[Drive] \nl
            Select a disk drive for output and read its directory.
\item[New] \nl
	    Create a new directory.  The name is entered on the prompt
	    line (i.e. the line below the menu).
\item[Output] \nl
            Mark the highlighted directory as the destination for
            all Write commands.  Make it the DOS working directory
            (that is, CD to it).
\item[Read] \nl
            Read the directory on the selected output (DOS) disk,
\item[Tag] \nl
            The Tag command marks the entire contents of the current
            directory, and all of its sub-directories, to be written
            out when the Write command is issued.
\item[Untag] \nl
            The Untag command removes the tag mark on the entire contents
            of the current directory and all of its sub-directories.
\item[Write] \nl
            Write all tagged files under the current directory. If the
	    current directory has any subdirectories containing tagged
files, corresponding subdirectories will be created in the output
directory.
\item[Quit] \nl
            The Quit command exits the program.
\item[/] \nl
            The '/' command returns control to the Main View.
\end{description}


\section{DOS File View}

\begin{description}
\item[ASCII, Binary] \nl
            Tells how to view the current file.
\item[Jcamp] \nl
            Mark the highlighted file as the Jcamp header file.
\item[Tag] \nl
            The Tag command marks current file to be written out
            when the Write command is issued.
\item[Untag] \nl
            The Untag command removes the tag mark on current file.
\item[View] ($\rightarrow$) \nl
            View a file in a window (non-ASCII files are viewed in Hex).
\item[Write] \nl
            Write the current file (whether or not it is tagged).
\item[Quit] \nl
            The Quit command exits the program.
\item[/] \nl
            The '/' command returns control to the Main View.
\end{description}


\section{ASCII File View}

\begin{description} 
\item[Binary] \nl
            The Binary command switches to the Binary File View.
\item[Quit] \nl
            The Quit command exits the program.
\item[/] \nl
            The '/' command returns control to the Main View.
\item[Esc] \nl
            The Escape key is equivalent to the Left Arrow command.

\end{description}


\section{Binary File View}

\begin{description}
\item[ASCII] \nl
            The ASCII command switches to the ASCII File View.
\item[Quit] \nl
            The Quit command exits the program.
\item[/] \nl
            The '/' command returns control to the Main View.
\item[Esc] \nl
            The Escape key is equivalent to the Left Arrow command.
\end{description}
 

 
\end{document}
