\documentstyle[sri]{book}
\pagestyle{headings}

\setlength{\unitlength}{1pt}
\thicklines

\title{{\bf RD3} \\ User's Guide} 
\author{Copyright \copyright\ 1988,1989 by Silvermine Resources, Inc.\\
	All rights reserved.}
%
% Commands local to this document
%
\newcommand{\OS}{PETOS}
\newcommand{\RD}{{\bf RD3}}

%%%%%%%%%%%%%%%%%%%%%%%%%%%%%%%%%%%%%%%%%%%%%%%%%%%%%%%%%%%%%%%%%%%%%%%%%%%%
 
\begin{document}

\titlepage
{
\setlength{\unitlength}{2pt}
\thicklines
\vspace{1in}
\begin{center}

\begin{picture}(140,85)
\put(1,1){\framebox(138,83){}}
\put(3,67){\framebox(134,15)[c]{\Huge RD3}}
\put(3,3){\framebox(134,61)[c]{\Huge User's Guide}}

\end{picture}

\copyright\ 1990 by Silvermine Resources, Inc.\quad
All rights reserved.

\end{center}
}


\tableofcontents
\markboth{}{Contents}

% === need blank page at end of TOC ===

\chapter{Introduction}
 
\RD\ is a program that runs on the IBM PC, XT, AT, and compatible
computers, for reading diskettes written under the \OS\
operating system, and transferring files from them to PC/MS-DOS
diskettes or hard disks.


\section{Features}
 
\begin{itemize}
\item
A display-oriented presentation of the directory structure of both
source and destination file systems.  The display includes multiple
views including side-by-side display of a disk's directory structure
(in tree form) and the contents of a selected directory in the tree.
 
\item
The ability to view the contents of files in either ASCII or
Hexadecimal on either the source or destination file system.
 
\item
Menus with extensive context-sensitive help.
 
\item
The ability to transfer an entire directory tree, preserving its
structure, with only two keystrokes.
 
\end{itemize}

\newpage

\chapter{Installation}
 
\section{Disk Contents}
The \RD\ disk contains the following files:
 
\begin{description}
\item[\code{RD3.EXE}] \nl
The executable file that implements the \code{rd3} command.  
 
\item[\code{MSHERC.COM}] \nl
A terminate-and-stay-resident (TSR) program required for displaying
graphics on a Hercules monochrome graphics card.

\item[\code{README.DOC}] \nl
A text file containing last-minute notes.

\item[\code{CSS\_HERC.DOC}] \nl
Information for people using \code{PECSS} with Hercules graphics
cards. 

\item[\code{AUTOEXEC.BAT}] \nl
A sample that illustrates how to set the \code{RD3} environment
variable. 

\item[\code{JCAMP.HDR}] \nl
Standard JCAMP-DX header file.

\item[\code{CMPRS.HDR}] \nl
JCAMP-DX header file for compressed data.

\item[\code{XYDATA.HDR}] \nl
JCAMP-DX header file for files with a single X-Y pair per line, for
convenient input into spreadsheet or database programs.

\end{description}
 
\newpage 
\section{Installing Compatibility Disk Drive}

\OS\ disks are written using single-density FM modulation, which a
PC cannot read.  In order to read \OS\ disks, you must install a
compatibility drive.  \RD\ is shipped with the MicroSolutions
Backpack drive, which comes with its own installation instructions.
You should install your compatibility drive, including its software
driver, before attempting to use \RD.

%=== may want more here ===

\section{Installing Files}

\begin{enumerate} 

\item
Start by making a {\em copy} of your \RD\ disk.
On a system with two floppy drives, give the command
\begin{dcode}
diskcopy a: b:
\end{dcode}
\noindent
and then put the \RD\ disk in drive \code{A:}, and a blank disk in
drive \code{B:}.
 
On a system with only one floppy drive, give the command
\begin{dcode}
diskcopy
\end{dcode}
\noindent
and then put the \RD\ disk in the floppy drive.  Exchange it with a
blank disk when the \code{diskcopy} program prompts you to do so.

\item 
Put the original disk in a safe place.

\item 
If you have a hard disk, make a directory on your working disk
(usually \code{C:}) called \code{rd3}, and copy all of the files on
the \RD\ disk to it.  You can do this with the commands:
\begin{dcode}
mkdir rd3\\
cd rd3\\
copy a:*.*
\end{dcode}
\noindent (assuming that the \RD\ disk is in drive \code{A:}).
 
(Strictly speaking, you don't have to do this.  You can put \RD\
in any directory that is listed in the \code{PATH} command in your
\code{AUTOEXEC.BAT} file.  We find it convenient, however, to keep
\RD\ and all of our \OS\ files in one place; the examples below
will assume this configuration.)
 
If you will be using floppy disks, you will probably find it best to
make another copy of the \RD\ disk and switch to it after booting
from your usual system disk.  It is also possible to put
\code{RD3.EXE} on your system disk and switch to a blank (but
formatted) DOS disk after starting \RD.  This leaves your entire
DOS disk free for transferring files.

\end{enumerate}

\section{Getting Started}

If you are on a two-floppy system, put the \RD\ working disk in either drive
\code{A:} or drive \code{B:}, and give the command \code{a:} or
\code{b:} respectively to make it your current working
drive. 

Then, give the command:
\begin{dcode}
rd3
\end{dcode}
\noindent
to start the \RD\ program. 
 
A menu is shown on the second line of the screen.  You can issue a
command from the menu either by typing the first character of the
command, or by moving the highlight to the command (\key{space} moves
forward and \key{backspace} moves backward) and hitting the \key{enter}
key.  The `?' key will {\em always} show you a screen with a
short summary of the current menu and its commands; the `/' key
always brings you back to the initial menu.  The \cmd{A}{About}
command in the initial menu gives you an overview of \RD\ and its
commands.
 
Try moving the highlighted menu item with the \key{space} and
\key{backspace} keys.  Notice that the line under the menu contains a
brief description of the highlighted item.  (The first word in the
menu is not a command but a label for the menu, and its help line
gives a general description of what commands in that menu are for.)

The line below the description line contains labels for the views
that occupy the lower portion of the screen.  On the left end of the
label line is the name of the view that will be reached by striking
the left arrow key (``From'') in the initial view, and on the right is
the name of the view that will be reached with the right arrow key.%
\footnote{In case you were wondering why double-angle symbols are
shown on the screen instead of arrows, it's because you might want to
use the ``Print Screen'' key, and the arrows correspond to control
characters that many printers can't handle.}  When there are two
views on the screen at the same time, the arrow labels will appear
over the ``active'' view, i.e. the one the menu refers to.

Next, try giving the commands \cmd{T}{To} (or right arrow) and
\cmd{R}{Read Disk}.  (Commands are single key\-strokes; no \key{enter}
key is needed, and either upper- or lower-case letters will work.)
You will end up with an outline-style display of your disk's directory
on the left, with the current directory hilighted, and a list of the
ordinary files in the current directory on the right.
 
Use the right arrow key to move the highlight into the File view. 
Highlight the file called \code{README} by moving down with the 
down arrow key.  Use the right arrow key again to view the contents
of the file.  Use the left arrow key twice to get back
into the Directory View.

Now go the the Drive View with the \cmd{D}{Drives} command.  Check to
see whether the compatibility drive (usually drive \code{E:}) has
been correctly identified.  \RD\ will usually identify it correctly,
but can be fooled by systems with removable hard drives, and by some
versions of the BIOS.

\begin{quote}
  If the compatibility drive is {\em not} correctly identified,
  specify the correct drive by using the up or down arrows to move to
  the correct drive, and giving the commands \cmd{S}{System} and
  \cmd{P}{PETOS}.  You may
  also have to specify the physical unit number with the
  \cmd{P}{Physical} command.  The Backpack drive is physical
  unit 32 by default.
\end{quote}

Now put a \OS\ disk in your compatibility drive (usually drive
\code{E:}) and give the command	\cmd{F}{From}  (If the drive is
already selected as the From drive, you can also use
left-arrow and \cmd{R}{Read}.  The PETOS disk's directory will be
read.  Notice that, because PETOS disks do not have subdirectories,
there is only one view for PETOS disks, combining the volume ID with
the files.


To transfer the entire contents of the \OS\ disk to the DOS disk,
simply give the commands \cmd{T}{Tag} and \cmd{W}{Write} with the
cursor highlighting the volume ID (the first line of the view).  If
you want to be more selective, you can highlight individual files and
tag them with the `T' command.  The `W' command in a File view
simply writes the highlighted file without tagging it; in a
Directory view (DOS only) it writes all tagged files in the highlighted
directory.

Note that by default, files are tagged as either ASCII or Binary.  
You can designate files as Binary (and tag them for transfer at the
same time) with the \cmd{B}{Binary} command.  Similarly the
\cmd{A}{ASCII} command tags files and designates them as ASCII. 
In \RD sp, \OS\ files with an extension of \code{.SP} are identified
as spectra and given a default conversion type which can be changed
either in the program (from the main view) or on the command line.
The default conversion type is ``S'' if not specified.
 

\chapter{Using RD3}
 
\section{Using the Menus}
 
Menus are shown on the second line of the screen.  You can issue a
command from the menu either by typing the first character of the
command, or by moving the highlight to the command (\key{space} moves
forward and \key{backspace} moves backward) and hitting the \key{enter}
key.  When using the first character of the command, only a single
keystroke is needed (no \key{enter} key), and either upper- or
lower-case can be used for letters.
 
To find out more about a menu command, use \key{space} and
\key{backspace} to move the highlight to that command.  The line
underneath the menu gives a brief description of the command.  To find
out more, type the `?' key, which will always show you a screen with a
short summary of the current menu and its commands; the `/' key always
brings you back to the initial menu.  The \cmd{A}{About} command in
the initial menu gives a more extensive overview of \RD\ and its
commands.


\section{Using the Cursor Keys}
 
The cursor keys are used for moving within a view, and between views. 
For moving within a view, the up and down arrows move a line at a
time, and the \key{PgUp} and \key{PgDn} keys move a screen-full at a
time.  The \key{Home} key moves to the first item in the view (i.e.
the root directory, first file in a directory, or first line in a
file).  The \key{End} key moves to the last item in the view.
 
The left and right arrow keys are used to move {\em between} views.
If there are two views on the screen, (e.g. the Directory and File
views), left and right arrows will move between them in the obvious
fashion.  Moving left from the Directory view goes to the {\em other}
Directory view (i.e. switches between the From and To views). 
The names of the views reached with the left and right arrows are
listed at the left and right edges of the screen (or active view, if
two views are on the screen), on the top line of the view frame.

Moving to the right always gets you more detail, and moving left less
(or something completely different, as in switching between Directory
Views).  Thus, moving right from the Directory view gets you to the
File view and moving right from the File view shows you the {\em
contents} of the highlighted file.
 
 
\section{Transferring Files}


Transferring files is a three-step process:

\begin{enumerate}
\item 
Select a destination directory for the files, using the
\cmd{O}{Output} command in the To Directory View.  Note that the
destination must be on a DOS disk.  If you do not select a
destination, the default is to transfer to the current working
directory ({\em not necessarily} the directory highlighted in the To
view, but the directory that was current when you started \RD; this
can sometimes be confusing if you highlight a directory and forget to
mark it as the destination).

Note that the current source disk and destination disk and the
current output directory are always shown on the bottom line of the
screen.

\item 
Tag some files to transfer.  The \cmd{T}{Tag} command tags the
highlighted file in the File View, and {\em everything} under the
highlighted directory in the Directory View.  The \cmd{A}{ASCII} and
\cmd{B}{Binary} commands in the File View both tag and specify a
file type for viewing.

You can assign a file type to everything under a directory using
the file type commands (A and B) in the Directory View.
In \RD sp,  the additional file type commands \cmd{C}{CSS},
\cmd{J}{JCAMP}, and \cmd{S}{Spectrum} are defined.  (Spectrum
specifies conversion to Perkin Elmer ``Data Manager'' format.)

The default conversion type for all spectrum files can also be
specified with the \cmd{C}{Conversion} command in the Main view. 

\item
Write the files with the \cmd{W}{Write} command in the
Directory View.

\end{enumerate}

\noindent
When transferring a single file the second step can be omitted by
issuing the \cmd{W}{Write} command in the File View with the
file you want to transfer highlighted.

%=== may not work in PETOS view ===

When a file is written its name may have to be changed; this is done
automatically.  A file's name may be changed in the following ways:
\begin{itemize}
\item
DOS filenames can contain only uppercase letters and a restricted set
of special characters.  Lowercase letters in the \OS\ filename are
uppercased, and special characters not permitted in DOS are replaced
by `\~{ }' characters.
\item
DOS filenames consist of an eight-character name, a period, and a
three-character extension.
\item
If the file to be written already exists, a number is appended to its
name (or replaces the last characters of the name, if the name is
already of maximum length).  The number is incremented until a name
conflict no longer occurs.
\item
If you decide that you need a new subdirectory to transfer your files
to, the \cmd{N}{New} command in the To Directory view lets you enter
the name of a directory to be created (under the currently-highlighted
directory).  The name is entered on the prompt line (the line just under
the menu); backspace can be used to correct mistakes.


\section{Transfer and Translation of Spectral Data}

Spectra recorded via CDS on the Perkin-Elmer 3x00 Data Stations are
saved with the .SP extension.  SP files are in binary form with a
header containing scan information, etc., and the data as binary
integers.

Spectral files on PC/MS-DOS systems can be in any of three formats,
none of which are compatible with the PETOS SP files.  For PECSS and
PE Data Manager(PEDM), the files are saved with a .SP extension but
the file formats are incompatible with each other.  Therefore,
depending upon whether you are using PECSS or PEDM, you should tag the
file in the PETOS view appropriately.  The conversion-type tag CSS
activates the PECSS translation.  The conversion-type tag SPECTRA
activates the translation to Data Manager format which is also used by
the PE 1600 series systems.

The third translation mode is JCAMP.  This mode translates the PETOS
SP file from binary to ASCII in a form which meets the JCAMP-DX
standard for data interchange.  You should use this translation for
transfer to any other DOS programs such as graphics or spreadsheets.
JCAMP translation requires that you give \RD\ the name of a header
file which adds information to the header of the PETOS SP file.  You
can specify the header file either in the environment variable
(\code{set RD3=}\ldots, see Section \ref{sec-cust}) or on the command
line.  It can also be specified or changed at any time by reading the
DOS directory containing the header file, highlighting your selection,
and pressing \cmd{H}{Header}.  The full pathname of the selected
header file is displayed in the upper right corner of your screen.

We have supplied three sample header files on your distribution disk.
You should feel free to modify these files to suit your particular
requirements.  In particular, you should place information in the
ORIGIN and OWNER lines.  The header files can contain any other
information as described in the JCAMP-DX specification.

Header files can also contain any of three directives to the JCAMP
translator. Each must be preceded by ``//'', must be in capitals, and
must be followed by ``='':

\begin{description}
\item[\tt //DECIMALS=n]\nl
          {\em Sets no. of decimal places in the ordinate, default=4.}
\item[\tt //DIF/DUP=]\nl
            {\em Outputs data in a compressed form.}
\item[\tt //XYDATA=]\nl
              {\em Outputs data in 2 columns: wavenumber and ordinate.
              This is most useful for spreadsheets, etc.}
\end{description}

The \cmd{C}{Conversion} selection on the main menu tells \RD\ to tag
all the \code{.SP} files for the desired translation as the directory
is read from the PETOS disk.  You can also set the default conversion in
the environment variable or the command line (see Section \ref{sec-cust}).

 
\section{Customization}
\label{sec-cust} 

\RD\ has several parameters that can be customized:  the default input
and output drives, the parameters of individual drives, and, in 
\RD sp the default conversion for spectrum files and the pathname of
the default JCAMP header file.
These can be specified in either of two ways: in the \RD\ command line,
or in your \code{AUTOEXEC.BAT} file.

To specify customization parameters in the command line,
follow the command \code{rd3} with one or more of the following
options:

\begin{description}

\item[\code{-i} \param{drive-number}] \nl
This specifies the input drive.  The drive should be specified
as a DOS drive letter followed by a colon.

\item[\code{-o} \param{drive-letter}] \nl
This specifies the output (DOS) drive.  The drive should be specified
as a letter followed by a colon.  An additional directory path may be
appended; it will be set as your current working directory.  This lets
you start up \RD\ and have it \code{cd} to whatever directory you
usually transfer files to.

\item[\code{-d} \param{drive-letter}{[HAP]}{[C]}{[=\param{phys-drive}]}] \nl
This specifies drive characteristics; any number of --d options can be
given.  The optional letter H, A, or P specifies a Hard drive,
AT-type (high-density) drive, or PC-type (low density) drive.  The
optional C designates a compatibility drive (Backpack or CompatiCard), 
which is required in order to read \OS\ disks.

\item[\code{-c} \param{conversion-type}] \nl
This specifies the default conversion type for spectrum (\code{.SP})
files; the \param{conversion-type} is one of the letters \cmd{C}{CSS},
\cmd{J}{JCAMP}, \cmd{S}{Spectrum}, \cmd{A}{ASCII}, or \cmd{B}{Binary}.
 
\end{description}
\noindent Note that a space is not required between the option
letter (i, o, p, or s) and the drive-letter, drive-number, or drive
parameter string that follows it.  Space is {\em prohibited} inside a
drive parameter string.

Customization parameters can also be specified by setting an
environment variable in your \code{autoexec.bat} file.  To this,
include a line in \code{autoexec.bat} of the form:
\begin{quote}
\code{set RD3 =} \param{options}
\end{quote}

\noindent where the options are the same as the command-line options
described above.

Placing an option into the command line overrides a corresponding
option in the SET command.

\chapter{Views and Operations}
 
 
This section contains a complete description of \RD's views and the
operations that can be performed in them (by means of both cursor keys
and menus).  It is a slightly expanded version of the online
help screens.

 
\section{Main View}
\begin{description}
\item[About]\nl
           This command gives you an overview of the program and
           how to use it.
\item[Conversion]\nl
           set the default Conversion type for .SP files.
\item[Drives]\nl
           Transfer control to the Drives view.
\item[Quit]\nl
           Exit the program.
\item[?]\nl
           Display help.
           Use the Esc or left arrow key to return from a help screen.
\end{description}

\section{Conversion Type Menu}
\begin{description}
\item[Ascii]\nl
	Mark files for ASCII transfer (applies {\em only} to files
	already in ASCII form).
\item[Binary]\nl
	Mark files for Binary (no) conversion on transfer.
\item[CSS]\nl
	Mark \code{.SP} files for CSS conversion.
\item[Jcamp]\nl
	Mark \code{.SP} files for Jcamp conversion (to be used
	whenever ASCII output is desired).
\item[Spectrum]\nl
	Mark \code{.SP} files for Spectrum conversion (use for PE Data
	Manager or PE 1600).
\item[Quit]\nl
           The Quit command exits the program.
\item[/]\nl
           The `/' command returns control to the Main View.
\item[Esc]\nl
           The Escape key returns to the Main View menu without making
           any selection.
\end{description}

\section{Drives View}
\begin{description}
\item[DriveType]\nl
           Select type of drive (PC, AT, etc.).  The default format
           of an  \OS\ disk is determined by the drive type.
\item[MediaType]\nl
           Select type (density) of media. \\
	   HD: 1.2Mb, MD: 720K, DD: 360K, LD: 180K
\item[System]\nl
           Select a file system (operating system) for the drive.
           The choices are DOS and \OS.
\item[From]\nl
           Specify hilited drive as source; go to From view.
\item[To]\nl
           Specify hilited drive as destination; go to To view.
\item[Quit]\nl
           The Quit command exits the program.
\item[/]\nl
           The `/' command returns control to the Main View.
\end{description}


\subsection{File System Type Menu}
\begin{description}
\item[DOS, PETOS]\nl
           Specify the file system (operating system) with which
           to read the current drive.  (PETOS applies only to the
           Compatibility drive.)
\item[/]\nl
           The `/' command returns control to the main menu.
\item[Esc]\nl
           The Esc key returns to the Directory View menu with no
           action.

\end{description}

\subsection{Drive Type Menu}
\begin{description}
\item[AT]\nl
           AT-type (high density).  Used for high-density DOS disks.
\item[Compatibility]\nl
           This selection is a toggle; it turns the Compatibility flag
           for the drive either on or off.  Compatibility drives are
	   required in order to read \OS\ disks.
\item[Hard]\nl
           Hard disk.
\item[PC]\nl
           PC-type ('double' density).
\item[Quit]\nl
           The Quit command exits the program.
\item[/]\nl
           The `/' command returns control to the Main View.
\item[Esc]\nl
           The Escape key returns to the Drive View menu without making
           any selection.

\end{description}

\section{Source (From) Directory View}
These commands also apply to the PETOS combined directory/file view
when the volume name (the top line of the view) is highlighted.

\noindent
\begin{description}
\item[ASCII, Binary, CSS, Jcamp, Spectrum]\nl
           Specify the conversion type on ALL files in the current
           directory and its subdirectories, and TAG the files for
           output as well.
\item[Drives]\nl
           Go to the Drives view to select a disk drive for input
           or change the parameters of the current input drive.
\item[Read]\nl
           Read the directory on the selected input disk,
\item[Tag]\nl
           The Tag command marks the entire contents of the current
           directory, and all of its sub-directories, to be written
           out when the Write command is issued.
\item[Untag]\nl
           The Untag command removes the tag mark on the entire contents
           of the current directory and all of its sub-directories.
\item[Write]\nl
           Write all tagged files under the current directory.
\item[Quit]\nl
           The Quit command exits the program.
\item[/]\nl
           The `/' command returns control to the Main View.
\end{description}


\section{Source File View}
These commands also apply to the PETOS combined directory/file view,
when a file (any line except the top line of the view) is highlighted.

\noindent
\begin{description}
\item[ASCII, Binary, CSS, Jcamp, Spectrum]\nl
           Tells what kind of conversion to do on the current file
           when it is written to the output directory.  (These also
           tag the file.)
\item[Tag]\nl
           The Tag command marks current file to be written out
           when the Write command is issued.
\item[Untag]\nl
           The Untag command removes the tag mark on the current file.
\item[View (or right arrow)]\nl
           View a file in a window (non-ASCII files are viewed in Hex).
\item[Write]\nl
           Write the current file (whether or not it is tagged).
\item[Quit]\nl
           The Quit command exits the program.
\item[/]\nl
           The `/' command returns control to the Main View.
\end{description}


\section{Destination Directory View}
\begin{description}
\item[Drives]\nl
           Go to the Drives view to select a disk drive for output
           or change the parameters of the current output drive.
\item[New]\nl
           Create a new directory.  Enter the name on the prompt line.
\item[Output]\nl
           Mark the highlighted directory as the destination for
           all Write commands.  Make it the DOS working directory
           (that is, CD to it) if it is on a DOS disk.
\item[Read]\nl
           Read the directory on the selected output disk,
\item[Tag]\nl
           The Tag command marks the entire contents of the current
           directory, and all of its sub-directories, to be written
           out when the Write command is issued.
\item[Untag]\nl
           The Untag command removes the tag mark on the entire contents
           of the current directory and all of its sub-directories.
\item[Write]\nl
           Write all tagged files under the current directory.
\item[Quit]\nl
           The Quit command exits the program.
\item[/]\nl
           The `/' command returns control to the Main Menu.
\end{description}


\section{Destination File View}
\begin{description}
\item[ASCII, Binary]\nl
           Tells how to view the current file.
\item[Header]\nl
	  Make this file the Jcamp header file
\item[Spectrum]\nl
	  Mark file for Spectrum viewing
\item[Tag]\nl
           The Tag command marks current file to be written out
           when the Write command is issued.
\item[Untag]\nl
           The Untag command removes the tag mark on the current file.
\item[View (or right arrow)]\nl
           View a file in a window (non-ASCII files are viewed in Hex).
\item[Write]\nl
           Write the current file (whether or not it is tagged).
\item[Quit]\nl
           The Quit command exits the program.
\item[/]\nl
           The `/' command returns control to the Main View.
\end{description}


\section{Ascii File View}
\begin{description}
\item[Binary]\nl
           The Binary command switches to the Binary File View.
\item[Quit]\nl
           The Quit command exits the program.
\item[/]\nl
           The `/' command returns control to the Main View.
\item[Esc]\nl
           The Escape key is equivalent to the Left Arrow command.
\end{description}


\section{Binary File View}
\begin{description}
\item[ASCII]\nl
           The ASCII command switches to the ASCII File View.
\item[Quit]\nl
           The Quit command exits the program.
\item[/]\nl
           The `/' command returns control to the Main View.
\item[Esc]\nl
           The Escape key is equivalent to the Left Arrow command.
\end{description}

 
\end{document}
        
